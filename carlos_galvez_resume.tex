%%%%%%%%%%%%%%%%%%%%%%%%%%%%%%%%%%%%%%%%%
% Plasmati Graduate CV
% LaTeX Template
% Version 1.0 (24/3/13)
% 
% Modified by: Carlos Galvez. Doesn't require XeLaTeX anymore
% This template has been downloaded from:
% http://www.LaTeXTemplates.com
%
% Original author:
% Alessandro Plasmati (alessandro.plasmati@gmail.com)
%
% License:
% CC BY-NC-SA 3.0 (http://creativecommons.org/licenses/by-nc-sa/3.0/)
%
% Important note:
% This template needs to be compiled with XeLaTeX.
% The main document font is called Fontin and can be downloaded for free
% from here: http://www.exljbris.com/fontin.html
%
%%%%%%%%%%%%%%%%%%%%%%%%%%%%%%%%%%%%%%%%%

%----------------------------------------------------------------------------------------
%	PACKAGES AND OTHER DOCUMENT CONFIGURATIONS
%----------------------------------------------------------------------------------------

\documentclass[a4paper,10pt]{article} % Default font size and paper size
\usepackage{fourier}

\usepackage[usenames,dvipsnames]{xcolor} % Required for specifying custom colors
\usepackage[utf8]{inputenc}

\usepackage{fullpage}
\usepackage[top=2.0cm,bottom=2.5cm,left=2.5cm,right=2.5cm]{geometry}

\usepackage{hyperref} % Required for adding links	and customizing them
\definecolor{linkcolour}{rgb}{0,0.2,0.6} % Link color
\hypersetup{colorlinks,breaklinks,urlcolor=linkcolour,linkcolor=linkcolour} % Set link colors throughout the document

\usepackage{titlesec} % Used to customize the \section command
\titleformat{\section}{\Large\scshape\raggedright}{}{0em}{}[\titlerule] % Text formatting of sections
\titlespacing{\section}{0pt}{3pt}{3pt} % Spacing around sections

\usepackage{array}
\newcolumntype{L}[1]{>{\raggedright\let\newline\\\arraybackslash\hspace{0pt}}m{#1}}
\newcolumntype{C}[1]{>{\centering\let\newline\\\arraybackslash\hspace{0pt}}m{#1}}
\newcolumntype{R}[1]{>{\raggedleft\let\newline\\\arraybackslash\hspace{0pt}}m{#1}}

\def \widthone {3.1cm}
\def \widthtwo {12.3cm}
\def \vspac {0.25cm}
\usepackage[backend=bibtex,isbn=false,doi=false]{biblatex}
\addbibresource{carlos_galvez_resume.bib}

\hyphenation{ca-meras si-mulation}

\begin{document}

\pagestyle{empty} % Removes page numbering

%----------------------------------------------------------------------------------------
%	NAME AND CONTACT INFORMATION
%----------------------------------------------------------------------------------------

\par{\centering{\Huge Carlos Gálvez}\bigskip\par} % Your name

\section{Personal Information}
\vspace{\vspac}
\noindent
\begin{tabular}{R{\widthone}|L{\widthtwo}}
\textsc{Date of Birth} & December 29\textsuperscript{th}  1991  \\
\textsc{Nationality} & Spanish \\
\textsc{Address} & Prinsgatan 11, Lgh 1401, 413 05 Gothenburg, Sweden\\
\textsc{Mobile Phone} & +46 72 966 9146\\
\textsc{E-mail} & \href{mailto:carlosgalvezp@gmail.com}{carlosgalvezp@gmail.com}
\end{tabular}


%----------------------------------------------------------------------------------------
%	WORK EXPERIENCE 
%----------------------------------------------------------------------------------------
\vspace{\vspac}
\section{Experience}
\vspace{\vspac}
\noindent
\begin{tabular}{R{\widthone}|p{\widthtwo}}
	\textsc{Apr} 2017 - \textsc{Present}& \textbf{Software Developer - Perception} at \textsc{Zenuity}, Sweden\\
	& Design and implementation of a fast prototyping vehicle platform to accelerate the development of self-driving cars. CAN/FlexRay/Ethernet vehicle interfacing.  Software containerization using Docker. Continuous integration and automatic testing of software components in virtual environments. DrivePX 2 setup and deployment. ROS-based real-time visualization. \\ 
	&\\
	\textsc{Aug} 2015 - \textsc{Apr} 2017& \textbf{Software Developer - Sensor Fusion} at \textsc{Volvo Car Corporation}, Sweden\\
	& Development of algorithms for sensor data fusion, including lidar, radar and camera, in the context of Volvo Cars' autonomous driving project \emph{Drive Me}. Experience in high-perfomance computing, safety-critical code as well as the ISO 26262 standard. Agile development and continuous integration workflow.  \\
	&\\
	%------------------------------------------------
	
	\textsc{Jun} 2014 - \textsc{Jul} 2014 & \textbf{Research Engineer} at \textsc{Computer Vision and Active Perception Lab}, KTH, Sweden\\
	& Development of an autonomous robot to perform 3D mapping with RGBD cameras in hardly accesible environments. Based on ROS, OpenCV and PCL.\\
	&\\
	
	% --------------------------------------------------
	
	\textsc{Oct} 2012 - \textsc{Oct} 2013 & \textbf{Fellowship} at \textsc{Signals and Systems Department}, ETSIT-UPM, Spain\\
	& Development of a vision-based parking occupancy estimation system, using OpenCV and Qt libraries. Involved in the national project \emph{Ciudad 2020}, related to smart cities. Scientific paper published at IET-ITS \cite{Galvez2015}. \\
	%&\\
	%% --------------------------------------------------
	%
	%\textsc{Oct} 2011 - \textsc{Oct} 2012 & \textbf{Fellowship} at \textsc{Electrical Engineering Department}, ETSIT-UPM, Spain\\
	%& Development and integration of a new educational hardware platform for the study of ARM microcontrollers at the Electronic Systems Laboratory.%\\
	
	% --------------------------------------------------
	%&\\
	%\textsc{Oct} 2009 - \textsc{Oct} 2010 &  \textbf{Fellowship} at \textsc{Telematic Engineering Department}, ETSIT-UPM, Spain\\
	%& Design of an optical handwritten character recognition system (OCR), with the aim of automatizing various teaching and administrative tasks.\\
	
\end{tabular}

%----------------------------------------------------------------------------------------
%	EDUCATION
%----------------------------------------------------------------------------------------
\vspace{\vspac}
\section{Education}
\vspace{\vspac}
\noindent
\begin{tabular}{R{\widthone}|p{\widthtwo}}	
\textsc{Oct} 2016 - \textsc{Present} & \textbf{Self-Driving Car Engineer Nanodegree} \\
& \textsc{Udacity}. \\
& Deep Learning, Computer Vision, Sensor Fusion, Localization and Mapping, Control.\\
& \\

%------------------------------------------------
\textsc{Aug} 2013 - \textsc{Jun} 2015 & \textbf{Systems, Control and Robotics, MSc} \\
& \textsc{KTH Royal Institute of Technology}, Stockholm, Sweden. \\ 
& Master's Thesis on Sensor Fusion for Autonomous Driving \cite{Galvez2015Thesis}.\\
& Advisor: Prof. John \textsc{Folkesson}, Examiner: Prof. Patric \textsc{Jensfelt}. \\
& \textsc{GPA}: A.\\
& \\
%------------------------------------------------

\textsc{Aug} 2013 - \textsc{Jun} 2015 & \textbf{Civilingenjörsutbilding, MSc Electrical Engineering} \\
& \textsc{KTH Royal Institute of Technology}, Stockholm, Sweden. \\
&\\

%------------------------------------------------

\textsc{Jul} 2014 - \textsc{Aug} 2014 & \textbf{Tohoku University Engineering Summer Programme (TESP)} \\
& \textsc{Tohoku University}, Sendai, Japan.\\
& Lectures and seminars related to Robotics. Project: lidar-based obstacle avoidance. \\
&\\

%------------------------------------------------

\textsc{Sep} 2009 - \textsc{Jun} 2015 & \textbf{Telecommunication Engineering, (5-year programme, MSc accredited by ABET)} \\
& \textsc{E.T.S.I. Telecomunicación, Universidad Politécnica de Madrid}, Spain. \\
& GPA: 9.20/10.0.\\

\end{tabular}


%----------------------------------------------------------------------------------------
%	PROJECTS
%----------------------------------------------------------------------------------------
\section{Projects}
\vspace{\vspac}
\noindent
\begin{tabular}{R{\widthone}|p{\widthtwo}}
\textsc{Jan} 2017 - \textsc{Feb} 2017& \textbf{Vision-based lane marking and vehicle detection for autonomous driving}.\\& Application of Computer Vision and Machine Learning techniques.\\
& \\

\textsc{Dec} 2016 - \textsc{Jan} 2017& \textbf{Behavioral cloning for autonomous driving}. \\ & Application of Convolutional Neural Networks to predict steering angle from images.\\
& \\

\textsc{Nov} 2016 - \textsc{Dec} 2016& \textbf{Traffic sign recognition using Deep Learning}.\\ & Achieved 95.8\% accuracy on test set.\\
& \\

\textsc{Mar} 2015 - \textsc{Jun} 2015 & \textbf{Face detector}.\\ & Image-based, combining Adaboost and Deep Learning. \\
& \\
\textsc{Oct} 2014 - \textsc{Dec} 2014 & \textbf{Maze exploration robot}.\\ & Control, 3D object recognition, mapping, localization and planning.\\
& \\

\textsc{Sep} 2012 - \textsc{Jan} 2013 & \textbf{Augmented Reality mobile application}.\\ & Real-time visual tracking and control of robots. \\
& \\


\textsc{Sep} 2011 - \textsc{Jan} 2012 & \textbf{Adversarial learning through genetic algorithms}. \\ & Predator-prey robot learning simulation.\\

\end{tabular} 

%----------------------------------------------------------------------------------------
%	SCHOLARSHIPS AND ADDITIONAL INFO
%----------------------------------------------------------------------------------------
\vspace{\vspac}
\section{Honours and Awards}
\vspace{\vspac}
\noindent
\begin{tabular}{R{\widthone}|L{\widthtwo}}
2015 & Winner of the robot contest for the course \emph{Robotics and Autonomous Systems}.\\
2009 - 2013 & Extraordinary Academic Performance Scholarship (Madrid Government).\\
2009 &  Highest Honours in High School. Best academic record (GPA: 10.0/10.0). \\
\end{tabular}

%----------------------------------------------------------------------------------------
%	LANGUAGES
%----------------------------------------------------------------------------------------
\vspace{\vspac}
\section{Languages}
\vspace{\vspac}
\noindent
\begin{tabular}{R{\widthone}|L{4.75cm} R{7cm}}
\textsc{Spanish:} & Mothertongue&\\
\textsc{English:} & Fluent &\emph{TOEFL iBT: 110/120, September 2012 (Spain)} \\
\textsc{Swedish:} & Advanced & \emph{CEFR: B2, June 2015 (Sweden)}
\end{tabular}

%----------------------------------------------------------------------------------------
%	COMPUTER SKILLS 
%----------------------------------------------------------------------------------------
\vspace{\vspac}
\section{Computer skills}
\vspace{\vspac}
\noindent
\begin{tabular}{R{\widthone}|L{\widthtwo}}
\textsc{Proficient} & C/C++, Python, Docker, \textsc{Matlab} \& \textsc{Simulink}, OpenCV, PCL, ROS.\\
\textsc{Intermediate} & Java, CMake, OpenGL, OpenCL, CUDA, Qt, Bash, Git, Gerrit, \LaTeX, Linux.\\
\textsc{Basic} & HTML, CSS, JavaScript, J2EE, SQL, Android, ASM, VHDL.\\
\end{tabular}\vspace{\vspac}
\noindent
GitHub repository: \href{https://github.com/carlosgalvezp}{https://github.com/carlosgalvezp}

%----------------------------------------------------------------------------------------
%	INTERESTS AND ACTIVITIES
%----------------------------------------------------------------------------------------
\vspace{\vspac}
\section{Interests}
\vspace{\vspac}
\noindent
Travelling, photography, hiking, cycling, reading, movies, music.  \\
\noindent
Learning through online courses (MOOC): Coursera, Udacity, edX, etc.

%----------------------------------------------------------------------------------------
%	PUBLICATIONS
%----------------------------------------------------------------------------------------
\vspace{\vspac}
\section{Publications}
\noindent
\begingroup
\renewcommand{\section}[2]{}%
%\renewcommand{\chapter}[2]{}% for other classes
\printbibliography[heading=bibintoc]
\endgroup
%----------------------------------------------------------------------------------------
\end{document}
